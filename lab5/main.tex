% this TeX file provides an awesome example of how TeX will make super 
% awesome tables, at the cost of your of what happens when you try to make a
% table that is very complicated.
% Originally turned in for Dr. Nico's Security Class
\documentclass[11pt]{article}
\usepackage{hyperref}

% Use wide margins, but not quite so wide as fullpage.sty
\marginparwidth 0.5in 
\oddsidemargin 0.25in 
\evensidemargin 0.25in 
\marginparsep 0.25in
\topmargin 0.25in 
\textwidth 6in \textheight 8 in
% That's about enough definitions
\usepackage{listings}

% multirow allows you to combine rows in columns
\usepackage{multirow}
% tabularx allows manual tweaking of column width
\usepackage{tabularx}
% longtable does better format for tables that span pages
\usepackage{longtable}
\usepackage{graphicx} %package to manage images
\graphicspath{ {images/} }
\begin{document}
% this is an alternate method of creating a title
%\hfill\vbox{\hbox{Gius, Mark}
%       \hbox{Cpe 456, Section 01}  
%       \hbox{Lab 1}    
%       \hbox{\today}}\par
%
%\bigskip
%\centerline{\Large\bf Lab 5: Security Audit}\par
%\bigskip
\author{Daniel Ocampo}
\title{Lab 5:Global Health Exploratory Data Analysis}
\maketitle

\section{Objectives}

To learn how to ask exploratory analytics questions which may be answered using an investigation
of global health data from the World Health Organization (WHO). To determine how global health
data can be manipulated to address specific types of questions. To create meaningful modifications
to the data for the production of plots to visually answer the questions.


\section{Questions For part 1}
\begin{enumerate}
    \item Does spending
more on health equate to longer
expected lifetimes?

\item What country has the best ratio of health expenditure to
expected lifetime?
 \item Are there
differences in
the trends between males
and females?
\item What is the
correlation between health
expenditure and suicide rate?

\item Have these trends changed significantly over the last
ten years?



\end{enumerate}

\section{ Part 2}




\begin{enumerate}



\item Your Health Data Analysis: In addition to the R code that provides your solutions, you
are to write a report that includes the following:

\begin{itemize}
\item The link and name of your WHO data set used for your exploratory questions.

\href{http://apps.who.int/gho/data/node.main.78?lang=en}{Health expenditure per capita, by country}

\href{http://apps.who.int/gho/data/view.main.SDG2016LEXv?lang=en}{Life expectancy
Data by country} 

\href{http://apps.who.int/gho/data/node.main.MHSUICIDEASDR?lang=en}{Suicide rates, age-standardized
Data by country} 

\item  Description of your data set (describe what each column contributes to the data set).
For this a simple sentence would be sufficient to introduce each column.\\
\textbf {Health expenditure per capita, by country}: Per capita total expenditure on health at average exchange rate (US\$)
This data set contributes the amount of money a person spends Per capita total expenditure on health at average exchange rate (US\$). This data set allows people to see how much they spent health expenses. 
\textbf{Age-standardized mortality rate (per 100 000 population}
This column gives of the years of all the suicide rate per 100,000 people. 

\textbf{Life expectancy at birth (years)}:
Life expectancy at birth reflects the overall mortality level of a population. It summarizes the mortality pattern that prevails across all age groups - children and adolescents, adults and the elderly.

\textbf{Life expectancy at age 60 (years)}:
Life expectancy at age 60 reflects the overall mortality level of a population over 60 years. It summarizes the mortality pattern that prevails across all age groups above 60 years.





* Does spending more on health equate to longer expected lifetimes?
  * This is our main question that we're interested in, and the others stem from
    this one.  We have heard that the United States spends more on healthcare
    than many other countries, yet has one of the least healthy populations, and
    we have heard that Japan spends the least and has one of the longest life
    expectancies in the world, and we're curious if the data supports those
    statements and what differences may lie in comparisons between countries.

* What country has the best ratio of health expenditure to lifetime expectancy?
  * As economics majors and minors, we're also interested in looking into which
    countries are getting the best "bang for the buck" when it comes to lifetime
    expectancies versus healthcare expenditures.  Once those countries are
    identified, numerous other research questions open up in the realm of what
    can the less efficient countries learn from the more efficient ones.

* What is the correlation between health expenditure and suicide rates?
  * An interesting side question we came up with, is the question of higher
    healthcare expenditures being potentially correlated with higher or lower
    average suicide rates.  This is a question that we don't have an expected
    outcome for, and the answer may very well be that they aren't related in any
    consistent way.  If there is a positive correlation, then perhaps this is
    evidence that more of that funding should be spent towards mental health.

* Are there differences in these trends between males and females?
  * Besides anatomy differences, males and females have historically tended to
    have very different occupations and workplaces, which might tilt the trends
    in one direction or another; looking at them separately might show some
    correlations where looking at them averaged out hid those correlations from
    recognition.

* Have these trends changed significantly over the last decade?
  * It is worth considering not only what the trends are with the most current
    data available, but also how they have changed, to see whether the trends
    stay the same, reverse, or shift.  Various factors such as advancements in
    medical treatments and changes in workplaces might have some bearing on the
    trends we're wanting to look at, and these factors take time to be reflected
    in the datasets.


\end{itemize}


\section{The Talk}
The talk was about data collection and how we should be careful with biases such as women lying about there weight, compared to male who are usually more open to them. How  surveys try to figure out if a person is lying by asking smart and effective question. Asking question that imply something but may not mean explicitly say something. Well also talked about data clustering, and how that is useful, and diff rent data sets. explanations why some people were getting more sick than others. Which was quiet interesting, cause one of her case studies was about how women in a coal industry were getting sick more than the men, well turned out that the women were washing the guys clothing and therefore just  get sick. That was one of explanation for the talk. It might have been a little diffrent but the idea is there. 

\end{enumerate}








\end{document}
