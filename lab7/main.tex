% this TeX file provides an awesome example of how TeX will make super 
% awesome tables, at the cost of your of what happens when you try to make a
% table that is very complicated.
% Originally turned in for Dr. Nico's Security Class
\documentclass[11pt]{article}


% Use wide margins, but not quite so wide as fullpage.sty
\marginparwidth 0.5in 
\oddsidemargin 0.25in 
\evensidemargin 0.25in 
\marginparsep 0.25in
\topmargin 0.25in 
\textwidth 6in \textheight 8 in
% That's about enough definitions

% multirow allows you to combine rows in columns
\usepackage{multirow}
% tabularx allows manual tweaking of column width
\usepackage{tabularx}
% longtable does better format for tables that span pages
\usepackage{longtable}

\begin{document}
% this is an alternate method of creating a title
%\hfill\vbox{\hbox{Gius, Mark}
%       \hbox{Cpe 456, Section 01}  
%       \hbox{Lab 1}    
%       \hbox{\today}}\par
%
%\bigskip
%\centerline{\Large\bf Lab 1: Security Audit}\par
%\bigskip
\author{Daniel Ocampo}
\title{Lab 7: Psychological analysis  Exploratory Data Analysis}
\maketitle

\section{Objective}
To explore statistical tools which are relevant for the evaluation of psychological data. In particular,
to be able to research how to use new R-statistics software packages and apply them to particular
contexts for which they were designed. To extract knowledge from the produced visualizations and
extracted interpretation of results.


\section{Part One:Obtain your data:}

Obtain your data: You may obtain your data from any online source as long as it is a
credible source and that the data stems from the psychology discipline. For an idea, you could
select the data available by the Cornell University website: https://www.ciser.cornell.
edu/ASPs/datasource.asp.


The data that I was able to find was from the Labor bureau statistics.\\
\href{https://www.bls.gov/cex/csxstnd.htm#2010}{BLS}. The reason I chose this data set was because of the wealth effects, I want to see based on the data if that is real. How much impact the financial crisis of 2008 has done to impact  the economy. 





\section{Part Two: Description of Data}

You are to write a short report to describe your data. Discuss what
the data contains and its purpose (i.e., why was this data collected, for what purpose?). You
may need to look this information up online from sources other than the one where you found
it.



I found this data based on the simple fact that is has so much information. One of thing that I find interesting  is how people spend there money. according to Hayek an economist people make rational decisions. The data wide range of data  from consumer spending, we currently have about a 3\% percent GDP growth. The united states has one of the strongest economies, and therefore I would like to see how the great economy works. If US has a good gdp what means the economy is doing very well.  The data contains all, how many homeowners vs renters and their age range. The data Also contains what products us consumers were spending on. It ranges to soap to diamonds. 



\section{part three:Tools and Software}

Using your credible data, you are to use the psych package to perform a
correlation analysis over variables that you will study in connection to your research questions.


library(psych)


\section{Part Four:Answerable Questions}


These are some of the first question that I came up with while doing my research in about consumer spending. 
\begin{itemize}
\item Does the wealth effect actually exist? 
\item Does Consumer spending reduce after an election
\item When are people more willing to buy commodities 
\item Does spending affect general life expectancy
\item Does greater population generate higher gdp?
\end{itemize}


\section{Part Five}

based on the data, when gdp is high more people are willing to buy homes, and therefore showing that they feel confident about the econmy. In the sense that the wealth effect does show how people feel about their wealth.  





\end{document}
